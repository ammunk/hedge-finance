% Created 2022-03-06 Sun 15:35
% Intended LaTeX compiler: pdflatex
\documentclass[11pt]{article}
\usepackage[utf8]{inputenc}
\usepackage[T1]{fontenc}
\usepackage{graphicx}
\usepackage{longtable}
\usepackage{wrapfig}
\usepackage{rotating}
\usepackage[normalem]{ulem}
\usepackage{amsmath}
\usepackage{amssymb}
\usepackage{capt-of}
\usepackage{hyperref}
\usepackage{svg}
\usepackage{nicefrac}
\author{Andreas Munk}
\date{\today}
\title{}
\hypersetup{
 pdfauthor={Andreas Munk},
 pdftitle={},
 pdfkeywords={},
 pdfsubject={},
 pdfcreator={Emacs 28.0.50 (Org mode 9.6)}, 
 pdflang={English}}
\usepackage{biblatex}
\addbibresource{~/.dotfiles/bibtex.bib}
\begin{document}

\begin{figure}[t]
\centering
\includesvg[scale=0.3]{t2k-logo}
\end{figure}

\section*{The T2K Token}
\label{sec:orgc938672}

The purpose of this document is to lay down the foundations for a future
business plan - specifically by defining the revenue stream for Token Toolkit,
both short and long term. This will make clear what role the T2K token plays in
our ecosystem.

The mission of Token Toolkit is to be a utility based cryptocurrency company
that develops valuable products of highest quality for the benefit of our
customers. As stated in our \href{https://docs.tokentoolkit.io/welcome/what-is-token-toolkit}{whitepaper} we seek to empower our customers with the
ability engage in the financial market by providing the necessary tools to do so
well below current market value. Tools otherwise usually locked behind massive
paywalls or reserved for ``the big boys'' (hedge funds and other big financial
institutions)\footnote{For more info see our \href{https://tokentoolkit.io/\#}{website} and \href{https://docs.tokentoolkit.io/welcome/what-is-token-toolkit}{whitepaper}}. This leads to Token Toolkit's revenue stream which we
divide into three substreams:

\begin{enumerate}
\item A subscription-like approach, involving staking a specific dollar(\$)-value
amount of our native T2K token in any one of our staking pools. This option
is our base product and unlocks the option to further. Token Toolkit profit
from the tokenomcis involved with buying and selling T2K.
\item Additional pay-per-use using our native T2K token. This grants access to all
other products not available using (1). \textbf{To use (2) customers must also use
(1)}. Token Toolkit profit from trading back into the liquidity pool the
paid T2K.
\item Pay-per-use of all products using fiat currencies with a 5\% ``fiat fee''.
\end{enumerate}

T2K, as such, serves as a \emph{proof of purchase}. The value of T2K lies within its
use to grant access to Token Toolkit's products. The astute reader may realized
that due to the fixed supply of T2K, it's monetary value will necessarily
fluctuate in response to supply and demand.

In order to ensure that you, our customer, always get what you pay for, whenever
you interact with any of our staking pools we take a snapshot. The snapshot
matches the T2K staked and their values against the dollar(\$) (specifically
BUSD) required to use our products. This means that once you stake an amount of
T2K that satisfy the staking requirements at that point in time, you will never
loose access to those services accompanying the staking pools. This is
\textbf{regardless} of the price movement of T2K. If, however, any amount of those T2K
is \textbf{withdrawn} we take a new snapshot and if the price of T2K has moved against
you, you will loose access to our services and must stake additional T2K to
regain access.

For instance, say the cost of using Token Toolkits base products is \$250. You
then buy the equivalent number of T2K (say \(x\)) on January 5th and immediately
stake those T2K. Token Toolkit takes a snapshot and records that on January 5th
you staked \(x\) T2K worth \$250 thereby granting you access to the base products.
Assume then you check up on your staked tokens on February 1st and that in the
meantime other customers have finished using Token Toolkit and decided to sell
back their T2K tokens. This cased the T2K value to drop. However, this has \textbf{no}
effect on your access to Token Toolkit's base products. As long as you never
unstake and thereby stake at least \(x\) T2K you maintain access to the base
product (you may, however, stake more T2K should you want to). Only if you
unstake \emph{any} amount of T2K leaving less than \$250 worth of T2K in the pool do
you loose access. On the other hand, should the value of T2K increase, you can
then unstake the surplus T2K (i.e. the number of T2K that is leaves your staked
amount worth more than \$250) and either sell those for profit or use them for
other Token Toolkit products.

What we want to emphasize is that T2K is not meant to be a speculative token.
However, due to its fixed supply there exists speculative opportunities as we
discuss in Section \hyperref[sec:orgfff7193]{What T2K is Not}. Speculation in the price of T2K due to
supply and demand comes at a high risk and it is not a game in which Token
Toolkit participate directly. Token Toolkit's revenue comes from selling our
services and using T2K as a proof of purchase.

\subsection*{Why Develop Our Own Token}
\label{sec:org19c6ce3}
The truth is that most crypto projects do not need a token whatsoever to carry
out their business. This is not to discount all the advantages the blockchain
technology brings. Rather it is to remind us that it ought to be absolutely
clear exactly what purpose any new token serves. While Token Toolkit could, in
principle, carry out business without T2K, there are two reasons why we decided
to make T2K,

\begin{enumerate}
\item It is first and foremost to support and utilize the blockchain technology. It
enables fast and easy purchasing without involving a third-party participant.
\item While one option is simply to accept transfers of BUSD it would have required
a different way of keeping track of who has paid for access to our services.
Using T2K alleviates the need to pair each customer with a username and
having them login every time they wish to use our products. By simply staking
T2K all we need is to connect to your crypto wallet of choice and that's it.
We need no additional information other than proof that you own sufficient
T2K - information provided by the blockchain.
\item Enables a burn mechanism of T2K which will increase the price floor of T2K.
This ensures that even by returning all T2K to the liquidity pool (every one
selling) the price of the T2K token would be greater than at launch. For
details see the [[*Burn
\end{enumerate}
Mechanism][Burn Mechanism]] section.

\section*{Pricing and Tokenomics}
\label{sec:orge58544d}

The price of Token Toolkit's product listed on the on Token Toolkit's \href{https://tokentoolkit.io/\#}{website}
will always show the BUSD price and the equivalent required T2K needed to
stake/pay. It is important to note that the listed BUSD/T2K requirement will not
match the BUSD/T2K price found on e.g. \href{https://pancakeswap.finance/}{pancakeswap} and other exchanges. This is
due to the tokenomics specified in the \href{https://docs.tokentoolkit.io/welcome/what-is-token-toolkit}{whitepaper} which dictate that of the 5\%
buy/sell tax 3\% goes towards the liquidity pool and 2\% goes towards Token
Toolkit. When Token Toolkit price their products the required T2K will take into
account those tokenomics. To illustrate, assume T2K trades at \(r=0.5
\nicefrac{BUSD}{T2K}\). Further, assume the staking requirement for Token
Toolkit's base product is \$250. The number of T2K required would be,

$$
N_{{\text{T2K}}} = \frac{250~\text{BUSD}}{r} \times c = \frac{250~\text{BUSD}}{0.5~\text{\nicefrac{BUSD}{T2K}}} \times c = 500~\text{T2K} \times c,
$$

where

$$
c= 0.95~\text{due to buy tax)}
$$

Therefore we have,

$$
N_{{\text{T2K}}} = 475~\text{T2K}
$$

The price discrepancy between the \(500\) T2K calculated just using the price \(r\)
and \(N_{\text{T2K}}=475~\text{T2K}\) is to account for the imposed buy tax.

Of the \$250, the profits made by Token Toolkit depends on whether you stake or
purchase a product using the acquired tokens. In the latter case, Token Toolkit
profits \(0.97\times \$250 = \$242.5\), where the remaining \$8.5 is the 3\% (buy
tax) that goes to the liquidity pool. In the former case Token toolkit profits
immediately \(0.02 \times \$250 = \$5\) (the 2\% buy tax). Once the customer
sells their tokens at a later time Token Toolkit profits again from the 2\%
marketing sell tax. Of course the equivalent dollar amount depends on how the
supply and demand has moved the T2K price.

As a general rule the customer should expect that Token Toolkit calculates the
price for staking as if the customer sells the required T2K immediately after
buying - i.e. Token Toolkit assumes the profits equal twice the 2\% tax i.e.
\(\$250 - \$250 \times 0.98^2 = \$9.9\). In practice this means that Token
Toolkit profits less if the price of T2K drops between the buy and sell and vice
versa.

In summary, if the price of a Token Toolkit product is \(p\) and assuming the
value of T2K stays constant between buy and sell, then Token Toolkit profits,

$$ \text{profit}= \begin{cases} p - p\times0.98^{2} & \text{if staking} \\ p\times0.97 & \text{otherwise} \end{cases}$$

\subsection*{Price Windows}
\label{sec:org5342354}
In order to ensure our customers gets a fair chance to buy the necessary T2K,
Token Toolkit will calculate the required T2K to match the set USD price of all
products at short random time intervals. This ensures that sudden price movement
of T2K will not prevent the costumer from carrying out intended purchases. These
\emph{price windows} are random such as to minimize potential abuse.


\section*{What T2K is Not}
\label{sec:orgfff7193}

Currently, too many projects exists solely for this purpose and hold no inherent
value. By amassing hopeful investors these projects ultimately rely on buying
pressure outperforming selling pressure. Eventually early investors cash out,
taking profit, which in turn leaves others at a loss. These projects boil down
to a game of money redistribution and the rules for determining the winners and
the losers are akin to those found in \emph{a game of chicken}. You do not have
search for long to recognize similarities between such projects and other more
scrupulous schemes.

T2K is not and should not be considered a speculative asset that falls in the
category described above. First of all our business model here at Token Toolkit
does not consider you, the buyer of T2K, an investor. You are our customer and
we provide you with a service. Our goal is and always will be to provide you
with products of sublime quality - products we ourselves use here at Token
Toolkit. However, while T2K serves as our \emph{proof of payment} mechanism, it
remains amenable to speculation and here is how. Speculation in T2K is analogous
to buying tickets to e.g. the Superbowl. You could choose to only buy the
Superbowl tickets you need, but if you expect the overall ticket supply to not
meet the high demand of the populace, you choose to buy an additional \(x=10\)
tickets early. If the initial offered tickets subsequently sell out but demand
remains, you can profit by selling the \(x\) tickets you bought previously. You
would expect to be able to sell at a price \(p_{\text{higher}}\) above the
initial market price \(p_{\text{init}}\), and your profits \(y\) would be the
difference \(y=x(p_{\text{higher}}-p_{\text{init}})\). Assuming
\(p_{\text{higher}}=\$110 > p_{\text{init}} = \$100\) that results in a net
profit of \(y=10\times (\$110 -\$100) = \$100\).

This is in fact what happened when Sony released their latest PS5 console. Due
to high demand and low supply, to buy a PS5 for 2021 Christmas you might have
had to give up between \$700-\$1000 - way above the \$500 retail price\footnote{\url{https://www.cbsnews.com/essentials/what-people-are-actually-paying-for-ps5s-this-christmas/}}. While
this may not appear particularly appealing to the general consumer, it is
nothing but the market forces responding to supply and demand. These mechanisms
are exactly what provide the speculative opportunities.

It is at your own risk to decide whether or not to partake in this speculative
game. At any rate, Token Toolkit as a company will never partake in this
speculative game. Specifically, Token Toolkit, as a company, do not and will
never own any T2K\footnote{Due to the nature of blockchain technology this cannot be programmatically enforced.}. Nonetheless, full disclaimer, the founders and
employees of Token Toolkit may receive T2K as part of their salary. This is to
allow for them to utilize Token Toolkit's products at their discretion.

\section*{Staking Pools and Voting Rights}
\label{sec:orga8e26a9}
There will be two different staking pools;

\begin{enumerate}
\item Liquidity providing staking pool
\item Utility staking pool without lockup. Staking will only happen if the amount
of T2K staked matches the BUSD required at the time of staking.
\end{enumerate}

Both pools give access to Token Toolkit's base products with no difference in price.

\section*{Liquidity Pool Tax and Staking Rewards}
\label{sec:org132e171}

There are two ways to introduce the 56\% outstanding T2K while keeping the effect
to the price of T2K at a minimum. (1) Burn them. Otherwise they will serve as a
sort of ``inflationary'' mechanism (2) Match the LP rewards as valued in BUSD with
T2K from the outstanding batch. This will effectively double the LP rewards
while any outstanding T2K remains.

Once all outstanding T2K are circulating the liquidity pool rewards defaults
back to the standard liquidity pool tax.

\section*{Burn Mechanism and Charity Pledge}
\label{sec:org459f81a}

As a way to raise the price floor of T2K and thereby indirectly further reward
the staking pools Token Toolkit pledges to burn 2\% of their net profits in T2K.

Further, while our products here at Token Toolkit's mainly revolve around
providing tools enabling profiting from engaging in the crypto market, we also
believe in sharing such profits with those less fortunate. Therefore, T2K pledge
give away initially 10\% of our net profits to charity.

\section*{NFTs}
\label{sec:org18771e4}

Token Toolkit will be creating different NFT that functions as fee waivers and
only a small amount will ever be created and distributed. Depending on the
particular NFT it can provide the owner free access to either all or a subset of
Token Toolkit's products.

These NFTs will either be auctioned off or given as part of marketing campaigns.

\section*{Products}
\label{sec:org4892258}

When customers connect their wallet to use Token Toolkit's products, our dapp
will execute a transaction process as follows. Prioritize transferring the
required T2K and in the case there is an insufficient amount of T2K the dapp
will then transfer the outstanding amount (after transferring T2K) in BUSD. If
the combined T2K and BUSD is insufficient the transaction will fail.

If a sufficient amount of combined T2K and BUSD is available the customer then
needs to accept the transaction via their crypto wallet.

\subsection*{Base Products and Utilities}
\label{sec:org6356dd3}

\begin{itemize}
\item Sniper bot V1: Executes bulk transactions with a significant chance the
transaction will fail depending on the target tokens preventative measures.
\item Sniper bot V2: Implements a queuing system based on a ``first-come-first-serve'' principle.
\begin{itemize}
\item Queue placement is available
\item As a safety measure; if the price of the sniped token increases too much
from the first transaction in the queue, all subsequent snipes will fail and the customer gets a refund.
\item Less likely to fail for those in front of the queue
\end{itemize}
\item Access to preprocessed and filtered blockchain data which Token Toolkit uses in-house.
\item Betting games. Betting with or against our price predictions models.
\end{itemize}

\subsection*{Payed Products}
\label{sec:org505338b}
\begin{itemize}
\item Accounting services: Calculate P/L of our customers for tax purposes
\item Sniper bot V3: Similar to V2 expect with auctioned queue placements.
Queue placement is determined by how much is payed in terms of faction of the
amount sniped up to a maximum of 10\%. Ties are resolved using the first
come, first serve principle.
\item Sniper bot V4: Set-and-forget. Functions similar to V3 bus also has built-in
trading strategies like trailing stop loss, limit orders etc
\begin{itemize}
\item Token Toolkit takes 5\% of net profit as a fee. When calculating the 5\% Token
Toolkit will charge fees after having deducted the amount paid for the
queuing placement.
\end{itemize}
\item Use of price prediction model for manual trading
\begin{itemize}
\item Base BUSD price
\end{itemize}
\item Asset and investment management using AI/ML.
\begin{itemize}
\item Price will be a \% of total profits made.
\end{itemize}
\end{itemize}

\subsection*{Other Product Ideas}
\label{sec:orgbea4273}
\begin{itemize}
\item Expanding our tools for the General Stock Market
\end{itemize}
\end{document}
